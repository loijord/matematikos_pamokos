\documentclass{article}
\usepackage[utf8]{inputenc}
\usepackage[L7x]{fontenc}
\usepackage[lithuanian]{babel}
\usepackage{lmodern}
\usepackage{amsmath}
\usepackage{graphicx}
\usepackage{verbatim}
\usepackage{hyperref}
\usepackage{tcolorbox}
\usepackage{mdframed}
\usepackage{tikz}
\usepackage{cancel}
\usepackage{color}
\usepackage[top=2cm, bottom=2cm, left=2cm, right=2cm, footskip=1cm, a4paper]{geometry}
\usepackage{hyperref}
\usepackage[upint]{stix}
\usepackage{amsthm}
\usepackage{indentfirst}
\usepackage{enumitem}
\usepackage{framed}
\usepackage{tasks}

\begin{document}
Mokyklinėje matematikoje sutinkami aritmetikos ir algebros žymėjimai atsirado renesanso laikotarpio Europoje, todėl pasirinkau apžvelgti renesanso laikotarpį. Tačiau kvadratinių lygčių sprendimo metodai, skaičių žymėjimas skaičių ašyje ir neigiamų skaičių savybės atrastos dar viduramžiais, ne Europoje. Apie viduramžių laikus plačiau \href{http://norvaisa.lt/matematika/mokykline-matematika/kodel-neigiamuju-skaiciu-sandauga-yra-teigiamas-skaicius/}{pasakoja Rimas}.

\section*{Matematikos istorijos apžvalga renesanso laikotarpiu}
\textbf{Naujų žymėjimų atsiradimas} - skaitome \href{http://jeff560.tripod.com/operation.html}{čia} ir \href{https://www.math.ucdavis.edu/~anne/WQ2007/mat67-Common_Math_Symbols.pdf}{čia}.

\textbf{1489m.} + ir - ženklai pirmąsyk pasirodo matematiniuose darbuose. Jie reiškia perviršį ir likutį.

\textbf{1515m.} del Ferro randa kubinės lygties sprendinį, tačiau jo sprendimas lieka niekam nežinomas.

\textbf{1525m.} Pirmąsyk panaudojamas simbolis $\sqrt{}$, reiškiantis kvadratinę šaknį.

\textbf{1535m.} Tartaglia taip pat randa kubinės lygties $x^3=ax+b$ sprendinį ir jį, paprašytas Kardano, atskleidžia su prašymu neviešinti informacijos, kol jis pats jos nepublikuos.

\textbf{1545m.} Pasirodo Kardano rašyta knyga \textit{Ars Magma}, turėjusi didelę istorinę svarbą ankstyvajame renesanse. Joje Kardanas paskelbia savo darbą su kitų tipų kubinėmis lygtimis (pvz. $x^3+ax^2=b$ ir pan.) pratęsdamas Tartaglios metodus. Sprendimą paskelbti Tartaglios metodus jis motyvavo tuo, kad jo taikytos formulės buvo jau žinomos del Ferro. Knygoje taip pat įtrauktas Kardano studento Ferrari sprendimas ketvirto laipsnio lygtims, kuriam būtina naudoti Tartaglia metodą. Visose lygtyse pagal ano meto matematiką leistina naudoti tik teigiamus koeficientus, nes neigiami skaičiai laikyti apgaulingais, tačiau Kardanas - pirmasis Europos matematikas, pasiūlęs, kad tie patys skaičiavimai gali būti atlikti su koeficientais nepriklausomai nuo jų ženklo. Taip pat jis pirmąsyk istorijoje panaudojo kompleksinius skaičius: vienoje knygos dalyje yra parodyta, kaip atlikti veiksmą $(5+\sqrt{-15})(5-\sqrt{-15})$. Kaip tuo metu būdavo užrašomi reiškiniai, galite pamatyti \href{http://www.ms.uky.edu/~sohum/ma330/files/eqns_2.pdf}{šičia, 4psl}. Sklaustai ligi 18 amžiaus matematikoje - retenybė.

\textbf{$\approx$ 1557-60m.} Matematiniuose darbuose pasirodo =, >,< ženklai vietoj frazių ,,lygu su'', ,,daugiau už'' ir ,,mažiau už'' vartojimo.

\textbf{1591m.} Vijetas parodo, kad kubinės lygties išsprendimo uždavinys yra ekvivalentus su antikiniu kampo dalijimo į 3 dalis uždaviniu. 

\textbf{1628m.} Matematiniuose darbuose pirmąsyk pasiūlomas modulio ženklas.

\textbf{1629m.} ir \textbf{1637m.} Nepriklausomuose Ferma ir Dekarto darbuose pirmąsyk aptariamas geometrinių kreivių naudojimas koordinačių sistemoje (algebra naudojama spręsti geometriniams uždaviniams ir atvirkščiai). Pavyzdžiui kubo padvigubinimo uždavinys ($x^3=2$) sprendžiamas sukertant parabolę $y=\frac{x^2}{2}$ ir hiperbolę $y=\frac{1}{x}$, o ketvirto laipsnio lygtis $x^4=ax^2+bx+c$ sprendžiama sukertant parabolę $y=x^2$ su parabole $x=\frac{y^2-ay-c}{b}$. Dar pusantro amžiaus lygčių sprendimas buvo paliktas algebrai, konstravimo uždaviniai - Euklido geometrijai, o algebrinė geometrija buvo teorija apie kreives.

\textbf{1629m.} Pasiūlyta naudoti šaknį su indeksu.

\textbf{1631m.} Matematiniuose darbuose pasirodo $\pm$ ženklas,

\textbf{1633m.} Matematiniuose darbuose pasirodo ,,:'' ženklas, naudojamas trupmenoms aprašyti.

\textbf{1634-37m.} Matematiniuose darbuose pasirodo dabartinis laipsnio žymėjimas.

\textbf{1637m.} Dekartas pasiūlo naudoti pilną šaknies žymėjimą ($\sqrt{\phantom x}$)

\textbf{1640m.} Ferma įrodo Mažąją Ferma Teoremą panaudodamas binominius koeficientus. Jis yra vienintelis renesanso laikų  mokslininkas, tyrinėjęs skaičių teoriją, tačiau daugelis jo prieitų rezultatų buvo gauti neaprašant naudojamų metodų. Jo išvadų ir hipotezių įrodymai buvo palikti Oileriui, Lagrandžui ir Ležanrui. 

\textbf{1654m.} Paskalis, nagrinėdamas binominių koeficientų savybes, savo darbuose pirmąsyk panaudoja dabartinę indukcijos (pradinis žingsnis - indukcinis žingsnis) konstrukciją, tačiau dar pora šimtų metų niekas nesuformuluoja sprendimuose naudojamo indukcinio samprotavimo sąvokos.

\textbf{1658m.} Matematiniuose darbuose pasirodo ,,÷'' ženklas, naudojamas dalinimo operacijai aprašyti.

\textbf{1684m.} Leibnicas panaudoja ,,:'' ženklą, reiškiantį ir trupmeną, ir dalinimo operaciją.

\textbf{1698m.} Savo laiške Paskaliui Leibnicas pasiūlo naudoti $\cdot$ ženklą.

\textbf{1733 m.} Į Lietuvą pakliūna pirmasis algebros vadovėlis \textit{Alpha matheseos}.

\textbf{1755m.} Oileris panaudoja sumavimo simbolį $\sum$.

\textbf{1761m.} Oileris panaudoja atvirštinio daugiklio sąvoką Mažosios Ferma Teoremos įrodyme.

\textbf{1796m.} Gausas sukonstruoja 17-kampį vien su apskritimais ir tiesėmis - figūrą, nenagrinėtą Senovės Graikų. Tai nulemia jo apsisprendimą tapti matematiku. Jis susidomi Oilerio, Lagrandžo ir Ležanro darbais ir jų neišspręstais uždaviniais. Dar po 5 metų jis išleidžia knygą \textit{Disquisitiones Arithmeticae}, kurioje vysto teoriją apie visus taisyklinguosius $n$-kampius. Ši knyga svarbi tuo, kad prisidėjo prie rimtesnių skaičių teorijos pokyčių ir moderniosios algebros atsiradimo, į kurią įeina abstrakčios daugianarių savybės. Į knygą taip pat įeina:
\begin{itemize}
\item įrodymas, kad kiekvieną skaičių galima išskaidyti pirminiais dauginamaisiais vieninteliu būdu;
\item apibrėžiama lyginio sąvoka ir pasiūloma ją panaudoti įsitiknant, kad visos daugianario $x^3-8x+6$ liekanos yra nenulinės dalijant iš 5.
\end{itemize}

\textbf{1837m.} Nežymus matematikas Wantzel pirmasis įrodo, kad kubo padvigubinimo ir kampo dalijimo į tris lygias dalis uždaviniai yra neišsprendžiami geometriniais konstravimais, papildo Gauso taisyklingųjų $n$-kampių teoriją ir patikslina (Dekarto darbe 1637m. tik numanomus) algebrinius sukonstruojamumo kriterijus.

\textbf{1838-39m.?} Dirichle apibrėžia algebrinio skaičiaus sąvoką ir ją naudoja plėtojant algebrinę skaičių teoriją.

\textbf{1857m.} Dedekindas pasiūlo vietoje lyginių (kongruentumo sąryšių) naudoti kongruentumo klases kaip algebrinius objektus.

\textbf{1858m.} Dedekindas pasiūlo iracionaliuosius skaičius aiškinti ne remiantis atkarpomis pagal Eudoxo (350m. p.m.) plėtotą proporcijų teoriją, o naudojant racionaliųjų skaičių aibes.

\textbf{1872m.} galiausiai Dedekindas pasiūlo realiųjų skaičių aibę tapatinti su skaičių tiese. Tai nepaprastos svarbos prielaida, nes geometrinę tiesę ir realiųjų skaičių aibė gali būti laikytini abipus vienareikšmėje atitiktyje. Savo prielaidoje Dedekindas naudoja tokius apibrėžimus: 
\begin{itemize}
\item iracionalusis skaičius $\lambda$ - tai racionaliųjų skaičių aibės $\mathbb{Q}$ padalijimas į du poaibius $L_\lambda$ ir $R_\lambda$, tokius, kad kiekvienas $L_\lambda$ narys yra mažesnis už $R_\lambda$ narį, poaibis $L_\lambda$ neturi didžiausio nario, o poaibis $R_\lambda$ neturi mažiausio nario. Negriežtai į skaičių $\lambda$ galime žiūrėti kaip į skylę tarp šių poaibių. 
\item Šiame apibrėžime išėmę sąlygą ,,$R_\lambda$ neturi mažiausio nario'' gauname realiųjų skaičių apibrėžimą.
\end{itemize}
Tolimesni uždaviniai suformulavus šią prielaidą yra apibrėžti realiųjų skaičių sąryšius ir operacijas naudojant aibes. Pavyzdžiui:
\begin{itemize}
\item Jei $a < b$, tai $L_a\subseteq L_b$ ir $L_a\neq L_b$
\item Jei $a\le b$, tai $L_a\subseteq L_b$
\end{itemize}

\textbf{1882m.} Lindermannas išsprendžia paskutiniąją iš trijų antikinių problemų - kvadrato, lygiapločio su skrituliu sukonstravimo klausimą. Jis parodo, kad skaičius $\pi$ ne tik negali būti gaunamas geometriniu konstravimo būdu, bet netgi negali būti kurio nors daugianario reikšmė, su kuria jis lygus 0, t.y. išreikštas radikalais.

\textbf{1888m.} Dedekindas apibrėžia uždarinio sąvoką ir ją panaudodamas apibrėžia natūraliuosius skaičius kaip aibės $\{0\}$ uždarinį virš operacijos ,,+1''. Ši natūraliųjų skaičių samprata yra būtina dabartiniam indukcijos modeliui apibrėžti. Ji lemia esminius pokyčius skaičių teorijoje, nes natūraliųjų skaičių operacijas + ir $\times$ jau galima apibrėžti induktyviai, be to, skaičių teorija tampa neatsiejama nuo aibių teorijos. 

\textbf{1890m.} Pirmąsyk panaudojamas $\subset$ simbolis.

\textbf{1917m.} Pirmąsyk mokyklose pasiūloma naudoti skliaustus, kad būtų išvengta nesusipratimų su operacijų eiliškumu.

\textbf{1921m.} Išleistas pirmasis algebros vadovėlis lietuvių kalba.
\end{document}